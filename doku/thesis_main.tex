\documentclass[12pt,oneside,titlepage]{scrartcl} 

\newcommand{\myAutor}{Rico  (Matrikelnummer 12345)\\ \> \> \> 
Henning   (Matrikelnummer 12345)\\ \> \> \> 
Julian  (Matrikelnummer 12345)} % Autor
\newcommand{\myAdresse}{Stra\ss e 123 \\ \> \> \> 57xxx Siegerland} % Adresse
\newcommand{\myTitel}{Browser RPG-Adventure} % Titel der Arbeit
\newcommand{\myBetreuer}{Daniel Bitzer} % Betreuer
\newcommand{\myLehrveranstaltung}{Web Technologie} % Lehrveranstaltung
\newcommand{\myMatrikelNr}{123456} % Matrikelnummer
\newcommand{\myOrt}{Siegen} % Ort
\newcommand{\myAbgabeDatum}{\today} % Datum der Abgabe
\newcommand{\mySemesterZahl}{5} % Semesterzahl
\newcommand{\myHochschulName}{FOM Hochschule für Oekonomie \& Management Essen} % Name der Hochschule
\newcommand{\myHochschulStandort}{Siegen} % Standort der Hochschule
\newcommand{\myStudiengang}{Wirtschaftsinformatik} % Studiengang
\newcommand{\myThesisArt}{Seminararbeit als Projektdokumentation} % Art der Arbeit
\newcommand{\myAkademischerGrad}{Bachelor of Science (B. Sc.)} % Zu erlangender akademische Grad
\newcommand{\myFirma}{Mustermann GmbH} % Firma

\usepackage[ngerman]{babel}
\selectlanguage{ngerman}
\usepackage[babel,german=quotes]{csquotes}

\usepackage[utf8]{luainputenc}
\usepackage[T1]{fontenc}
\usepackage{fancyhdr}
\usepackage{fancybox}
\usepackage[a4paper, left=4cm, right=2cm, top=4cm, bottom=2cm]{geometry}
\usepackage{graphicx}
\usepackage{colortbl}
\usepackage[capposition=top]{floatrow}
\usepackage{array}
\usepackage{float}      %Positionierung von Abb. und Tabellen mit [H] erzwingen
\usepackage{footnote}
\usepackage[singlelinecheck=false, labelfont=bf, font=bf]{caption} % Tabellendesign lt. Leitfaden
\usepackage{caption}
\usepackage{enumitem}
\usepackage{amssymb}
\usepackage{mathptmx}
\usepackage[scaled=0.9]{helvet} 
\usepackage{courier}
\usepackage{amsmath}
\usepackage[table]{xcolor}
\usepackage{marvosym}			% Verwendung von Symbolen, z.B. perfektes Eurozeichen
\usepackage[colorlinks=true,linkcolor=black]{hyperref}
\definecolor{darkblack}{rgb}{0,0,0}
\hypersetup{colorlinks=true, breaklinks=true, linkcolor=darkblack, menucolor=darkblack, urlcolor=darkblack}
\renewcommand\familydefault{\sfdefault}
\usepackage{ragged2e}

\usepackage[hang, multiple]{footmisc} % Mehrere Fussnoten nacheinander mit Komma separiert
\setlength{\footnotemargin}{1em}

%\usepackage{todonotes} % todo Aufgaben als Kommentare verfassen für verschiedene Editoren

\usepackage{epstopdf} %Pakete für Tabellen
\usepackage{nicefrac} % Brüche
\usepackage{multirow}
\usepackage{rotating} % vertikal schreiben
\usepackage{mdwlist}
\usepackage{tabularx}% für breitenangabe

\definecolor{dunkelgrau}{rgb}{0.8,0.8,0.8}
\definecolor{hellgrau}{rgb}{0.0,0.7,0.99}
% Colors for listings
\definecolor{mauve}{rgb}{0.58,0,0.82}
\definecolor{dkgreen}{rgb}{0,0.6,0}

\usepackage{listings} % sauber formatierter Quelltext
% JavaScript als Sprache definieren:
\lstdefinelanguage{JavaScript}{
	keywords={break, super, case, extends, switch, catch, finally, for, const, function, try, continue, if, typeof, debugger, var, default, in, void, delete, instanceof, while, do, new, with, else, return, yield, enum, let, await},
	keywordstyle=\color{blue}\bfseries,
	ndkeywords={class, export, boolean, throw, implements, import, this, interface, package, private, protected, public, static},
	ndkeywordstyle=\color{darkgray}\bfseries,
	identifierstyle=\color{black},
	sensitive=false,
	comment=[l]{//},
	morecomment=[s]{/*}{*/},
	commentstyle=\color{purple}\ttfamily,
	stringstyle=\color{red}\ttfamily,
	morestring=[b]',
	morestring=[b]"
}

\lstset{
	%language=JavaScript,
	numbers=left,
	numberstyle=\tiny,
	numbersep=5pt,
	breaklines=true,
	showstringspaces=false,
	frame=l ,
	xleftmargin=5pt,
	xrightmargin=5pt,
	basicstyle=\ttfamily\scriptsize, 
	stepnumber=1,
	keywordstyle=\color{blue},          % keyword style
  	commentstyle=\color{dkgreen},       % comment style
  	stringstyle=\color{mauve}         % string literal style
}

\usepackage[
    style           = apa6, 
    uniquelist      = false,    
    maxcitenames    = 6,
    backend         = biber,
	urldate         = short,
	uniquename 		= true,
	language		= ngerman
	]{biblatex} 
		
	%% START Block für Funktion (1. Nennung von 2-6 Autoren: Alle Namen, danach nur noch 1. Name + et.al) 
	\usepackage{lmodern} 
	\makeatletter
	\newcommand{\apamaxcitenames}{6}
	\DeclareNameFormat{labelname}{%
	  \ifthenelse{\value{uniquelist}>1}
		{\numdef\cbx@min{\value{uniquelist}}}
		{\numdef\cbx@min{\value{minnames}}}%
	  \ifboolexpr{test {\ifnumcomp{\value{listcount}}{=}{1}}
				  or test {\ifnumcomp{\value{listtotal}}{=}{2}}}
		{\usebibmacro{labelname:doname}%
		  {\namepartfamily}%
		  {\namepartfamilyi}%
		  {\namepartgiven}%
		  {\namepartgiveni}%
		  {\namepartprefix}%
		  {\namepartprefixi}%
		  {\namepartsuffix}%
		  {\namepartsuffixi}}
		{\ifboolexpr{test {\ifnumcomp{\value{listtotal}}{>}{\apamaxcitenames}}
					 or test {\ifciteseen}}
		 {\ifnumcomp{\value{listcount}}{<}{\cbx@min + 1}
		   {\usebibmacro{labelname:doname}%
			 {\namepartfamily}%
			 {\namepartfamilyi}%
			 {\namepartgiven}%
			 {\namepartgiveni}%
			 {\namepartprefix}%
			 {\namepartprefixi}%
			 {\namepartsuffix}%
			 {\namepartsuffixi}}
		   {}%
		  \ifnumcomp{\value{listcount}}{=}{\cbx@min + 1}
			{\ifnumcomp{\value{listcount}}{<}{\value{listtotal}}
			  {\printdelim{andothersdelim}\bibstring{andothers}}
			  {\usebibmacro{labelname:doname}%
				{\namepartfamily}%
				{\namepartfamilyi}%
				{\namepartgiven}%
				{\namepartgiveni}%
				{\namepartprefix}%
				{\namepartprefixi}%
				{\namepartsuffix}%
				{\namepartsuffixi}}}
			{}%
		  \ifnumcomp{\value{listcount}}{>}{\cbx@min + 1}
		   {\relax}%
		   {}}%
		 {\usebibmacro{labelname:doname}%
		   {\namepartfamily}%
		   {\namepartfamilyi}%
		   {\namepartgiven}%
		   {\namepartgiveni}%
		   {\namepartprefix}%
		   {\namepartprefixi}%
		   {\namepartsuffix}%
		   {\namepartsuffixi}}}}
	\makeatother 
	\DeclareLanguageMapping{ngerman}{ngerman-apa}
	%% ENDE Block für Funktion (1. Nennung von 2-6 Autoren: Alle Namen, danach nur noch 1. Name + et.al)
	
	\DeclareDelimFormat*{finalnamedelim}{\addspace\bibstring{and}\space} % In Parencite von "&" auf "und" ändern
	\hypersetup{hidelinks}  %Grüne Links auf Literaturverz. unterdrücken.
	\setlength\bibitemsep{1.3ex} % Abstände im Literaturverzeichnis erhöhen
	\setlength\bibnamesep{1.0ex}
	\AtBeginBibliography{\singlespacing} % Zeilenabstand im Literaturverzeichnis ist Einzeilig - siehe Leitfaden S. 14
	\urlstyle{same} %Standard-Font für Link anstelle der "Schreibmaschinenschrift"
	\DeclareFieldFormat[misc]{urldate}{[#1\printfield{urldate}].} 	% Anpassung @online + @misc Bibl: Datum in eckigen Klammern ans Ende
	\DeclareFieldFormat[misc]{title}{\mkbibemph{#1}} %Anpassung @online + @misc Titel Kursiv	
	\DeclareFieldFormat[misc]{url}{Verfügbar unter\space \url{#1}} 	%Anpassung @online + @misc Text vor URL
	\renewbibmacro*{url+urldate}{\usebibmacro{url}\setunit{\addspace}\usebibmacro{urldate}}  % URL vor Abrufdatum setzen + getrennte Wörter "Verfügbar" und "Unter" entfernen
%%%% APA ENDE

\addbibresource{literatur.bib} %Bib-Datei einbinden
%\nocite{*} % Die folgende Zeile trägt ALLE Werke aus literatur.bib in das Verz. ein, auskommentieren um nur die anzuzeigen, die zitiert wurden


\usepackage{hyphsubst} %Silbentrennung
\HyphSubstIfExists{ngerman-x-latest}{%
\HyphSubstLet{ngerman}{ngerman-x-latest}}{}

\graphicspath{{./}{./media/}} % Pfad fuer Abbildungen

\usepackage{titletoc} % Weitere Ebene einfügen
\makeatletter

% Setze die Tiefe des Inhaltsverzeichnis auf 4 Ebenen -  Damit erscheinen \paragraph-Sektionen auch im Inhaltsverzeichnis
\setcounter{secnumdepth}{4}
\setcounter{tocdepth}{2}

% Fuege Abstand nach unten wie in einer normalen \section hinzu
\renewcommand{\paragraph}{%
  \@startsection{paragraph}{4}%
  {\z@}{3.25ex \@plus 1ex \@minus .2ex}{1.5ex plus 0.2ex}%
  {\normalfont\normalsize\bfseries\sffamily}%
}

\makeatother

\usepackage{appendix} % Paket für die Nutzung von Anhängen
\usepackage{setspace} % Zeilenabstand 1,5-zeilig
\onehalfspacing

\setlength{\parindent}{0mm} % Absätze durch eine neue Zeile
\setlength{\parskip}{0.8em plus 0.5em minus 0.3em}

\sloppy					%Abstände variieren
\pagestyle{headings}

\usepackage[printonlyused]{acronym} % Abkürzungsverzeichnis


% PDF Meta Daten setzen
\hypersetup{
    pdfinfo={
        Title={Browser RPG-Adventure},
        Subject={\myStudiengang},
        Author={\myAutor},
        Build=1.1
    }
}

% Umlaute in Code korrekt darstellen - siehe  https://en.wikibooks.org/wiki/LaTeX/Source_Code_Listings
\lstset{literate=
	{á}{{\'a}}1 {é}{{\'e}}1 {í}{{\'i}}1 {ó}{{\'o}}1 {ú}{{\'u}}1
	{Á}{{\'A}}1 {É}{{\'E}}1 {Í}{{\'I}}1 {Ó}{{\'O}}1 {Ú}{{\'U}}1
	{à}{{\`a}}1 {è}{{\`e}}1 {ì}{{\`i}}1 {ò}{{\`o}}1 {ù}{{\`u}}1
	{À}{{\`A}}1 {È}{{\'E}}1 {Ì}{{\`I}}1 {Ò}{{\`O}}1 {Ù}{{\`U}}1
	{ä}{{\"a}}1 {ë}{{\"e}}1 {ï}{{\"i}}1 {ö}{{\"o}}1 {ü}{{\"u}}1
	{Ä}{{\"A}}1 {Ë}{{\"E}}1 {Ï}{{\"I}}1 {Ö}{{\"O}}1 {Ü}{{\"U}}1
	{â}{{\^a}}1 {ê}{{\^e}}1 {î}{{\^i}}1 {ô}{{\^o}}1 {û}{{\^u}}1
	{Â}{{\^A}}1 {Ê}{{\^E}}1 {Î}{{\^I}}1 {Ô}{{\^O}}1 {Û}{{\^U}}1
	{œ}{{\oe}}1 {Œ}{{\OE}}1 {æ}{{\ae}}1 {Æ}{{\AE}}1 {ß}{{\ss}}1
	{ű}{{\H{u}}}1 {Ű}{{\H{U}}}1 {ő}{{\H{o}}}1 {Ő}{{\H{O}}}1
	{ç}{{\c c}}1 {Ç}{{\c C}}1 {ø}{{\o}}1 {å}{{\r a}}1 {Å}{{\r A}}1
	{€}{{\EUR}}1 {£}{{\pounds}}1 {„}{{\glqq{}}}1
}

\pagestyle{fancy} % Kopfbereich / Header definieren
\fancyhf{} % Seitenzahl oben, mittig, mit Strichen beidseits: \fancyhead[C]{-\ \thepage\ -}

\fancyhead[C]{\thepage} % Seitenzahl oben, mittig, entsprechend Leitfaden ohne Striche beidseits
\renewcommand{\headrulewidth}{0.4pt} % Waagerechte Linie unterhalb des Kopfbereiches anzeigen. Alternativ weg: \renewcommand{\headrulewidth}{0pt}

\begin{document} % Start the document here

\pagenumbering{Roman}								% Seitennumerierung auf römisch umstellen
\renewcommand{\refname}{Literaturverzeichnis}		% "Literatur" in "Literaturverzeichnis" umbenennen
\newcolumntype{C}{>{\centering\arraybackslash}X}	% Neuer Tabellen-Spalten-Typ: Zentriert und umbrechbar

\begin{titlepage} % Titlepage/Deckblatt
	\newgeometry{left=2cm, right=2cm, top=2cm, bottom=2cm}
	\begin{center}
		\textbf{\myHochschulName}\\
		\textbf{Hochschulzentrum \myHochschulStandort}\\
		\vspace{1.5cm}
			\includegraphics[width=3cm]{media/fomLogo} \\
		\vspace{1.5cm}
		Berufsbegleitender Studiengang\\
		\myStudiengang, \mySemesterZahl. Semester\\
		\vspace{2cm}
		\textbf{\myThesisArt}\\
		%\textbf{zur Erlangung des Grades eines	}\\
		%\textbf{\myAkademischerGrad}\\
		% Oder für Hausarbeiten:
		\textbf{im Rahmen der Lehrveranstaltung}\\
		\textbf{\myLehrveranstaltung}\\
		\vspace{2cm}
		über das Thema\\
		\Large{\myTitel}\\
		\vspace{0.2cm}
	\end{center}
	\normalsize
	\vfill
	\begin{tabbing}
		Links \= Mitte \=Mittez \= Rechts\kill
		Betreuer: \> \> \>\myBetreuer\\
		\> \> \\
		Autoren: \> \> \> \myAutor\\
		%\> \> \>  Matrikelnummer: \myMatrikelNr\\
		%\> \> \> \myAdresse\\
		\> \> \>  \\
		Abgabe: \> \> \> \myAbgabeDatum
	\end{tabbing}
\end{titlepage}

%-------Ende Titelseite-------------


% Sperrvermerk
%\newpage
%\thispagestyle{empty}
%\section*{Sperrvermerk}
%Die vorliegende Abschlussarbeit mit dem Titel \enquote{\myTitel} enthält unternehmensinterne Daten der Firma \myFirma . Daher ist sie nur zur Vorlage bei der FOM sowie den Begutachtern der Arbeit bestimmt. Für die Öffentlichkeit und dritte Personen darf sie nicht zugänglich sein.
%\par\medskip
%\par\medskip
%\_\_\_\_\_\_\_\_\_\_\_\_\_\_\_\_\_\_\_\_\_\_\_\_ \hspace{1.5cm} \_\_\_\_\_\_\_\_\_\_\_\_\_\_\_\_\_\_\_\_\_\_\_\_ \\
%(Ort, Datum)\hspace{4.5cm}
%(Eigenhändige Unterschrift)
%\newpage

% 
% Inhaltsverzeichnis
\setcounter{page}{2}
\addtocontents{toc}{\protect\enlargethispage{-20mm}}
\clearpage
\tableofcontents
\newpage
\setcounter{tocdepth}{2} %wurde  in zusaetzlichesMaterial.tex auf 0 gesetzt um Inhalt des Anhangs zu verbergen. Dadurch gehen allerdings Abbildungs und Tabellenverzeichnis kaputt.

%\addcontentsline{toc}{section}{Abbildungsverzeichnis} % Falls das Abkürzungsverzeichnis im Inhaltsverzeichnis angezeigt werden 
%\section*{Abbildungsverzeichnis} % Abbildungsverzeichnis
\listoffigures % Abbildungsverzeichnis
\newpage

%\addcontentsline{toc}{section}{Tabellenverzeichnis} % Falls das Abkürzungsverzeichnis im Inhaltsverzeichnis angezeigt werden 
%\section*{Tabellenverzeichnis} % Tabellenverzeichnis
\listoftables % Tabellenverzeichnis
\newpage

%\addcontentsline{toc}{section}{Abkürzungsverzeichnis} % Falls das Abkürzungsverzeichnis im Inhaltsverzeichnis angezeigt werden soll einkommentieren.
\section*{Abkürzungsverzeichnis} % Abkürzungsverzeichnis

\begin{acronym}[WoWi]\itemsep0pt %der Parameter in Klammern sollte die längste Abkürzung sein. Damit wird der Abstand zwischen Abkürzung und Übersetzung festgelegt
	\acro{PoC}{Proof of Concept}
	\acro{WoWi}{Wohnungswirtschaft}
\end{acronym}
\newpage

\pagenumbering{arabic} % Seitennummerierung auf arabisch und ab 1 beginnend umstellen
\setcounter{page}{1}










%-----------------------------------
% Abschnite der Arbeit
%----------------------------



\section{Aufgabenbeschreibung}

(TODO!): Rico schreibt das.

Umfang 1-3 Seiten


\subsection{Was ist das Ziel der Projektarbeit? Worin bestehen die (wahrscheinlichen) Herausforderungen? (allg. technisch und auch persönlich)}

Primär: 
Fertiges, spielbares Programm 
Projektablauf erfolgreich gestalten 
Web-Technologien nutzen
Fokus auf die technische Umsetzung, Gameplay sekundär

Sekundär: 
Lernkurve
Projektmanagement leben
Erfahrungen im Spieldesign 

Herausforderungen (siehe \ref{2021-11-27-projektskizze-2})

\begin{itemize}
    \item techn. Umsetzbark.
    \item Zeit
    \item Know How
    \item Fokusverlust
    \item geeignete Aufgabefelder für eine Aufteilung finden 
\end{itemize}





\newpage


\section{Anforderungen}

(TODO!): Henning schreibt das.


\subsection{Welche Techniken/ Technologien sollen eingesetzt werden, um die Aufgabe zu lösen/ realisieren? Warum sollen gerade diese eingesetzt werden? Gibt es besondere Anforderungen? (technisch, Benutzer, sonstige)}


Django gewählt, da Team in Web-Projekten /-entwiclungen unerfahren, Django weiter für die Anforderungen geeignet erschien und wir es als Teil der Vorlesungen im Modul kennenglernt haben. 

Alternative Frameworks sind zwar vom Namen bekannt, wurden aber weiter nicht genauer in betracht gezogen. Auch damit grundsätzliche Erfahungen im Umgang mit HTML, CSS und JavaScript im Rahmen des Projkets durch das Team gesammelt werden können. Die Abkürzung über ein Framework wurde daher hier auch bewusst nicht gewählt. 




Allg. Anforderunegn:
Rahmen ist gesetzt: Web-App.



Besondere Anforderungen: 

Gesetzt durch Dozenten: 
\begin{itemize}
    \item Charakterentwicklung
    \item Klassen
    \item Mulitplayerkomponente
\end{itemize}







\newpage


\section{Herangehensweise}
Bereits in der frühen Konzeptphase des Projektes, entschieden wir unsere Arbeit mithilfe des Projektmanagement-Modells nach Scheurer zu organisieren (\textcite{scheurer}). Da dieses umfangreiche Theorie, Werkzeuge und Best Practices für Großprojekte bereitstellt, entschieden wir weiterhin nur die für unser vergleichsweise kleines Softwareprojekt sinnvollen Teile zu implementieren.
In ersten informellen Brainstormings zu grundlegenden Themen wie Spieldesign, Gameplay-Mechaniken, Setting und Technik wurden die groben Rahmenbedingungen der Entwicklung abgesteckt. Diese wurden dann in einer \enquote{Kick-Off}-Veranstaltung konkretisiert und in einer detaillierten Projektsizze festgehalten (im Folgenden zu sehen in den Abbildungen 1 a) und b)).
%TO-DO: Abbildungen in Anhang + Referenzierungen
\begin{figure}[H]
    \centering
    \label{fig:2021-11-27-projektskizze}
    \caption{Projektsizze vom 27.11.2021}
    \subfloat[][]{\fbox{\includegraphics[width=0.5\linewidth]{2021-11-27-projektskizze-1}}}
    \subfloat[][]{\fbox{\includegraphics[width=0.5\linewidth]{2021-11-27-projektskizze-2}}}
\end{figure}
%Projektskizze updaten -> Meilensteine rein, nicht realisierte Parts raus
Ziel einer solchen Skizze ist es, einen fokussierten Überblick über alle Aspekte des durchzuführenden Projektes zu gewinnen. Neben allgemeinen Informationen wie Projektname und Auftraggeber, werden Zielsetzung, Aufgaben, erwartete Ergebnisse, Risiken und Randbedingungen definiert. Auch werden wichtige Termine hervorgehoben und Meilensteine für einzelne Teiltätigkeiten gesetzt. Der Definition einer konkreten Zielsetzung kommt hierbei eine herausragende Bedeutung zu, da sich viele der nachfolgenden Aspekte aus eben dieser ergeben. 
\newpage
Den ersten Meilenstein des Projektes stellte die Entwicklung eines rudimentären Prototypen dar, der die technische und konzeptionelle Machbarkeit des Spieles demonstrieren sollte. Darauf aufbauend konnte dann eine (grobe) Aufwandsschätzung erfolgen und es wurden weitere Meilensteine gesetzt.

In Anbetracht unserer eigenen Zielvorstellungen und der im obigen Kapitel erläuterten globalen Anforderungen, entschieden wir das Projekt in drei nebenläufige Teilprojekte zu unterteilen: 
\begin{itemize}
    \item Infrastruktur und Backend-Entwicklung (Henning Beier)
    \item Frontend-Entwicklung und Grafik-Design (Julian Schäfer)
    \item Game-Design (Rico Pursche)
\end{itemize}

Im Rahmen der \enquote{Infrastruktur- und Backend-Entwicklung} sollte das technische Grundgerüst des Spiels erstellt werden. In diesen Bereich fielen Aufgaben wie die Bereitstellung der Server-Architektur, die Definition und Verwaltung der Datenbank und die Programmierung der Game-Logik.

Im Teilprojekt \enquote{Frontend-Entwicklung und Grafikdesign} lagen die Gestaltung der Website mit HTML, CSS und JavaScript im Fokus. Ein weitere Aufgabe war die Erstellung sämtlicher Grafik-Assets, d.h. Concept Art, Bilder, Sprites, Icons und Animationen.

Setting, Gameplay-Konzept, das Schreiben von Texten, die Ausarbeitung des Kampfsystems und das damit verbundene Balancing lagen im Aufgabenbereich des Teilprojektes \enquote{Game-Design}.

Diese Aufteilung ist hierbei nicht als strikte Trennung zu verstehen, bei der die einzelnen Arbeitsbereiche voneinander abgeschottet sind. Vielmehr wurden Verantwortungsbereiche abgesteckt, die sich auch überschneiden können und das in der Umsetzung auch taten. So wurde häufig in \enquote{teilprojektfremden} Tätigkeitsbereichen gearbeitet, was aber von Beginn an so vorgesehen war. Der Austausch von Daten, Informationen und die Versionsverwaltung wurde über ein gemeinsames Github-Repository realisiert. Änderungen wurden lokal getestet und dann in das Repository geschoben. 

Zudem wurde ein regelmäßiges Meeting am Sonntag eingeplant. Hier wurden der allgemeine Projektstatus besprochen, die Einhaltung der gesetzten Meilensteine überprüft und gemeinsame Entscheidungen über verschiedenste Aspekte des Projektes getroffen.
Ebenfalls wurden Konzepte für Teilthemen wie z.B. Frontend-Layout, Setting und viele Weitere erarbeitet und aufeinander abgestimmt (in den folgenden Abbildungen beispielhaft zu sehen).

\begin{figure}[H]
    \centering
    \caption{Frühes UI-Layout Mockup}
    \label{fig:2021-12-05-Projketbesprechung-Miro-d}
    \fbox{\includegraphics[width=0.7\textwidth]{2021-12-05-Projketbesprechung-Miro-d}}
\end{figure}


\begin{figure}[H]
    \centering
    \caption{Konzept: Kampfsystem vom 29.11.2021}
    \label{fig:2021-11-29-konzept_kampfsystem}
    \fbox{\includegraphics[width=0.7\textwidth]{2021-11-29-konzept_kampfsystem}}
\end{figure}






\newpage



\section{Vorstellung des Ergebnisses}

(TODO!): Julian schreibt die Übersicht hierzu.

Einleitung mit eher detailiertem Überblick über das Spiel.

Gerne alle Funktionen benennen, nichts erklären. Überblick verschaffen. 


Dazu Spielablaufskizze abarbeiten (Die Zeichnung(\ref{fig:2022-01-08-Spielablauf-Chart}) und Erläuterung erklären)

\begin{figure}[H]
    \centering
    \caption{Diagramm: Seitenaufbau und Spielablauf}
    \label{fig:2022-01-08-Spielablauf-Chart}
    \includegraphics[width=1\textwidth]{2022-01-08-Spielablauf-Chart}
\end{figure}


(keine 5 Seiten)


\newpage


\section{Reflektion}

An dieser Stelle finden sich stichpunktartig einige Reflektionen von uns, die sich zum Abschluss der Arbeit ergaben und aus unserer Sicht hier durchaus nennenswert sind. 

    \textbf{Rundenbasierter vs. chaotischer Spielablauf:} Der Ansatz die Entwicklung durch den Einsatz eines rundenbasierten Ablaufs bzw. Kampfsystems deutlich zu vereinfachen, musste im Laufe der Entwicklung zumindest angezweifelt werden. Denn der Aufwand, der durch die notwendige Konzeptionierung und Detailplanung entsteht, ist nicht zu unterschätzen. Dementgegen stünde bei einem chaotischen Spielablauf lediglich das Handling der Events. 

    \textbf{Kleinteilige Aufgabenpakete:} In der Entwicklung eine überschaubare Anzahl an kleinteiligen bzw. Teilaufgaben vor sich zu haben, empfanden wir als sehr hilfreich. Man hat damit einen Überblick über die Arbeit der nächsten Tage. Bei der Entwicklung der Rundenlogik entstand durch die Kenntniss um die nächsten Rundenschritte hier ein stets guter Überblick. Der Abschluss jeder einzelnen Teilaufgabe sorgte weiter laufend für positive Motivation.

    \textbf{Lernkurve und Codequalität:} Eigentlich müsste am Ende eines Entwicklungsprojektes stets noch einmal von vorne anfangen. Allein um alles zu korrigieren und anzupassen bzw. auf einen gleichen Quaitätsstand zu bringen, was im Projektverlauf bei den Beteiligten an Fähigkeiten und Wissen gelernt wurde.

    \textbf{Setting:} Rückwirkend betrachtet, scheint die Wahl des Settings, abhängig von Spielmechanismen, die nachher nicht umgesetzt wurden, nicht mehr so entscheidend, wie zu dem Zeitpunkt der Entwicklung. Denn so wie das Spiel jetzt aufgebaut ist, hätte es durchaus auch als Science Fiction funktionieren können. Es müssten lediglich der optische Auftritt und die Geschichten abgeändert werden. Der technische Unterbau, speziell der des Kampfmechanismus, welcher mitentscheidend für die Änderung war, ist zum gegenwertigen Zeitpunkt sehr universell und einfach gehalten und könnte so ohne Änderungen übernommen werden.

    \textbf{Storytelling/Szenenentwicklung:} Damit ist das Team im Großen und Ganzen sehr zufrieden. Die Zusammenarbeit zwischen Autor und Artist war engmaschig und hat gut funktioniert. Es war jedoch nicht immer einfach, nicht zu weit abzuschweifen oder die Szene zu komplex werden zu lassen, denn der Autor hätte sich manchmal noch mehr Tiefe gewünscht, um all seine Ideen umzusetzen. Außerdem merkt der Spieler im fertigen Spiel nicht, welcher Aufwand betrieben wurde, um die Texte und die Szenen zu entwickeln. Für das anvisierte Ziel, welches zu Anfang des Projektes gesteckt wurde, nämlich technische Umsetzbarkeit und Implementierung aller Anforderungen, war der Umfang vielleicht jetzt schon etwas viel. Denn der Aufwand, im Besonderen im Grafikdesign, war schon recht hoch.

    \textbf{Balancing:} Das Balancing funktioniert grundsätzlich gut. Die Tatsache, dass der Charakter seine Erfahrung getrennt für HP und AP ausgeben und dadurch eine extreme Schere zwischen den Charakteren entstehen kann, macht das Balancing aber auch recht schwer. Hier wäre es vielleicht mit einem Ansatz, mit festen Erfahrungspunkten je Gegner und ein starres Levelsystem, bei dem man bei festgelegten Grenzen einen Level aufsteigt und feste HP und AP Zuwächse je Level bekommt, einfacher gewesen. Dadurch ließen sich die Gegner in den aufeinander folgenden Leveln viel besser skalieren, gleichzeitig würde aber die Variabilität und Individualität der Charaktere stark beschnitten werden.
    
    \textbf{Grafiken \& Animationen:} Der Arbeitsaufwand für die Erstellung der Intro- und Hintergrundgrafiken und die Animation der Gegner, wurde von uns zu Beginn der Entwicklung stark unterschätzt. Selbst ohne die Zeit für die Bewältigung von einhergehenden Problemen, wie z.B. dem Erlernen halbwegs vorzeigbarer Zeichentechniken, mitzuzählen, waren viele Stunden notwendig, um alle Assets fertigzustellen. Auf der anderen Seite machen aber gerade Grafikdesign und -stil einen sehr großen Teil des Charmes und der Identität eines Spieles aus. Es ist nun aber besser nachzuvollziehen, warum sich bei der Entwicklung von Spielen durch professionelle Studios ganze Abteilungen um nichts anderes kümmern als um Concept Art, Animation und Grafik-Design.
    
    \textbf{Verwendung eines CSS-Frameworks:} Zu Anfang der Entwicklung hatten wir angedacht benötigte Assets für die Frontend-Entwicklung vollständig selbst zu erstellen. So war auch die grafische Aufbereitung von HTML-Elementen zunächst etwas, was wir selbst machen wollten. Hier wurde jedoch schnell klar, dass dies einen enormen Aufwand notwendig machen würde und den Rahmen des verhältnismäßig kleinen Projektes sprengen würde. Das hervorragende CSS-Framework RPGUI konnte an dieser Stelle sehr gewinnbringend eingesetzt werden und hat uns beim Frontend-Design sehr geholfen.
    
    \textbf{Genereller Lernkurve:} Vor diesem Projekt verfügten die beteiligten Studenten über keine oder nur sehr geringe Erfahrung im Bereich der Webentwicklung. Weder die Abläufe einer Spielentwicklung noch das Django-Framework waren uns näher bekannt. Da am Ende ein einfaches, aber funktionierendes Spiel dabei herausgekommen ist, kann man behaupten, dass die Lernkurve aller Projektmitarbeiter entsprechend steil gewesen ist. Wir werden sicherlich einiges an Erfahrungen und Best Practices aus diesem Projekt für die Zukunft mitnehmen können.   







%-----------------------------------
% Apendix / Anhang
%-----------------------------------
\newcommand{\AppendixName}{Anhang}
\newpage
\section*{\AppendixName} %Überschrift "Anhang", ohne Nummerierung
\addcontentsline{toc}{section}{\AppendixName} %Den Anhang ohne Nummer zum Inhaltsverzeichnis hinzufügen

\begin{appendices}
% Nachfolgende Änderungen erfolgten aufgrund von Issue 163
\makeatletter
\renewcommand\@seccntformat[1]{\csname the#1\endcsname:\quad}
\makeatother
\addtocontents{toc}{\protect\setcounter{tocdepth}{0}} %
	\renewcommand{\thesection}{\AppendixName\ \arabic{section}}
	\renewcommand\thesubsection{\AppendixName\ \arabic{section}.\arabic{subsection}}
	


In diesem Abschnitt sind aktuell einige Unterlagen eingefügt, die im Rahmen des Projektes eine Relevanz hatten. Vor Abgabe der Projektarbeit, soll dieser Abschnitt überarbeitet, ausgedünnt und ergänzt werden. 

\section{Projektnotizen}

Austausch und Zusammenarbeite erfolgte auf verschiedenen Platformen:
\begin{itemize}
    \item Gezeichnet und Entwürfe wurden meist in Miro erstellt: \url{https://miro.com/app/board/uXjVOdN2haQ=/}. 
    \item Besprechungen erfolgten meist in Teams: \url{https://teams.microsoft.com/l/team/19%3aDoBvOwOIC6WNhsL9kOIYFKNtVftU1yBtcEn_gcyQtcg1%40thread.tacv2/conversations?groupId=850a22ff-34a2-4fe2-a506-f55ac4d595f8&tenantId=b9b6f99a-a243-422d-ab36-f726574c981a}. 
    \item Der gemeinsame Code und die Dokumentation wurden auf Github erstelt: \url{https://github.com/tstsrv-de/tstsrv-de}. 
\end{itemize}

\subsection{Projektbesprechungen}

Stets Sonntags erfolgten Projektbesprechungen. Notizen und Zusammenfassungen davon finden sich hier in fortschreitender, chronologischer Reihenfolge. Ebenso hier entsprechend einsortiert, finden sich Konzeptzeichnungen und Entwürfe aller Art (UI, Code, Datenbankmodelle).

(TODO!) Bildbeschreibungen ergänzen, wichtige Bilder beschreiben.

2021-11-23-erster-entwurf-gameloop 
\begin{figure}[H]
    \centering
    \caption[]{2021-11-23-erster-entwurf-gameloop}
    \label{fig:2021-11-23-erster-entwurf-gameloop}
    \includegraphics[width=1\textwidth]{2021-11-23-erster-entwurf-gameloop}
\end{figure}

2021-11-23-erstes-db-konzept 
\begin{figure}[H]
    \centering
    \caption[]{2021-11-23-erstes-db-konzept}
    \label{fig:2021-11-23-erstes-db-konzept}
    \includegraphics[width=1\textwidth]{2021-11-23-erstes-db-konzept}
\end{figure}

2021-11-27-erstentwurf-ui 
\begin{figure}[H]
    \centering
    \caption[]{2021-11-27-erstentwurf-ui}
    \label{fig:2021-11-27-erstentwurf-ui}
    \includegraphics[width=1\textwidth]{2021-11-27-erstentwurf-ui}
\end{figure}

2021-11-27-projektskizze-1 
\begin{figure}[H]
    \centering
    \caption[]{2021-11-27-projektskizze-1}
    \label{fig:2021-11-27-projektskizze-1}
    \includegraphics[width=1\textwidth]{2021-11-27-projektskizze-1}
\end{figure}

2021-11-27-projektskizze-2 
\begin{figure}[H]
    \centering
    \caption[]{2021-11-27-projektskizze-2}
    \label{fig:2021-11-27-projektskizze-2}
    \includegraphics[width=1\textwidth]{2021-11-27-projektskizze-2}
\end{figure}

2021-11-29-Entwurf-Klassen-Ui 
\begin{figure}[H]
    \centering
    \caption[]{2021-11-29-Entwurf-Klassen-Ui}
    \label{fig:2021-11-29-Entwurf-Klassen-Ui}
    \includegraphics[width=1\textwidth]{2021-11-29-Entwurf-Klassen-Ui}
\end{figure}

2021-11-30-Entwurf-Lobby-Logik 
\begin{figure}[H]
    \centering
    \caption[]{2021-11-30-Entwurf-Lobby-Logik}
    \label{fig:2021-11-30-Entwurf-Lobby-Logik}
    \includegraphics[width=1\textwidth]{2021-11-30-Entwurf-Lobby-Logik}
\end{figure}

2021-11-30-Entwurf-Lobby-UI 
\begin{figure}[H]
    \centering
    \caption[]{2021-11-30-Entwurf-Lobby-UI}
    \label{fig:2021-11-30-Entwurf-Lobby-UI}
    \includegraphics[width=1\textwidth]{2021-11-30-Entwurf-Lobby-UI}
\end{figure}

2021-12-02-Countdown-Logik 
\begin{figure}[H]
    \centering
    \caption[]{2021-12-02-Countdown-Logik}
    \label{fig:2021-12-02-Countdown-Logik}
    \includegraphics[width=1\textwidth]{2021-12-02-Countdown-Logik}
\end{figure}

2021-12-05-Projketbesprechung-Miro-b 
\begin{figure}[H]
    \centering
    \caption[]{2021-12-05-Projketbesprechung-Miro-b}
    \label{fig:2021-12-05-Projketbesprechung-Miro-b}
    \includegraphics[width=1\textwidth]{2021-12-05-Projketbesprechung-Miro-b}
\end{figure}

2021-12-05-Projketbesprechung-Miro-c 
\begin{figure}[H]
    \centering
    \caption[]{2021-12-05-Projketbesprechung-Miro-c}
    \label{fig:2021-12-05-Projketbesprechung-Miro-c}
    \includegraphics[width=1\textwidth]{2021-12-05-Projketbesprechung-Miro-c}
\end{figure}

2021-12-05-Projketbesprechung-Miro-d 
\begin{figure}[H]
    \centering
    \caption[]{2021-12-05-Projketbesprechung-Miro-d}
    \label{fig:2021-12-05-Projketbesprechung-Miro-d}
    \includegraphics[width=1\textwidth]{2021-12-05-Projketbesprechung-Miro-d}
\end{figure}

2021-12-11-Projekt-Besprechung-Klassenbeschreibung 
\begin{figure}[H]
    \centering
    \caption[]{2021-12-11-Projekt-Besprechung-Klassenbeschreibung}
    \label{fig:2021-12-11-Projekt-Besprechung-Klassenbeschreibung}
    \includegraphics[width=1\textwidth]{2021-12-11-Projekt-Besprechung-Klassenbeschreibung}
\end{figure}


\begin{figure}[H]
    \centering
    \caption[]{11.12.2021: Projekt Besprechung: Gegenseitiges Update und Wechsel von Szenenlogik zu Kampfsystem für das RPG }
    \label{fig:2021-12-11-Projekt-Besprechung}
    \includegraphics[width=1\textwidth]{2021-12-11-Projekt-Besprechung}
\end{figure}



\section{Entwicklungsnotizen}

\subsection{Entwicklung, Sonntag 19.12.2021:}

An diesem Tag wurden weitere, notwenige Grundlagen für die Integration der Spiellogik eingebaut. Das insbesondere in Vorbereitung auf die kommenden Anpassungen und Entwicklungen die in der Projektbesprechung vom 11.12.2021 besprochen wurden. 
Konkret: 

\begin{itemize}
	\item Prüfung auf den Seiten Chars, Worldmap und Lobby ob dieser Benutzer ein aktives Spiel hat. Falls ja, wird der Benutzer auf diese Seite umgeleitet.
	\item Grundfunktion für das Beenden von einem Spiel eingebaut: Man kann nun per Klick im Spiel, das Spiel beenden.
	\item Daran anschließend eine Prüfung im laufendem Spiel, ob das Spiele beendet wurde und falls ja, Anzeige eines Endbildschirms.
\end{itemize}

Die Entwicklung der Grundlagen an diesem Tag wurde mit Fokus auf Modularisierung erledigt. Der Code der jeweiligen Funktionen wurde in einzelnen Dateien ausgelagert um Wiederverwendbarkeit und Lesbarkeit zu erhöhen. 

Die zugehörigen Commits sind insbesondere: 
\url{https://git.io/JDjKl} und 
\url{https://git.io/JDjKB}


\subsection{Entwicklung, Dienstag 21.12.2021:}

Grundlagen des Kampfsystems entsprechend der Projekt-Besprechung vom 11.12.2021 (Abbildung \ref{fig:2021-12-11-Projekt-Besprechung}) sollen implementiert werden.

Vorbereitungen: 

\begin{itemize}
    \item Tabelle "GamesScenesSteps" und Verknüpfungen entfernen (Commits \url{https://git.io/JDjKz} und \url{https://git.io/JDjKg})
    \item Kampf-/Gamelog erzeugen: Darin werden alle Meldungen aus dem Spiel wie z.B. Kampftexte, Schaden, Aktionen, Systemmeldungen und alles andere denkbare angezeigt und gespeichert. Getrennt davon soll der Chat-Log dargestellt werden. Dazu werden in der Tabelle "Games" zwei neue Textfelder erzeugt (Commit \url{https://git.io/JDj63}). 
    \item Anzeige des Game-Logs auf der Spielseite. Schreiben von Nachrichten in das Gamelog als ersten Test des grundlegend umgestellten Seitenaufbaus: Es werden nur noch einzelene Elementinhalte per Websocket transportiert, nicht mehr ganze HTML-Code-Blöcke (Commit \url{https://git.io/JyeJQ}).
\end{itemize}


\subsection{Entwicklung, Montag 27.12.2021:}

Weitere Entwicklungen entsprechend der Projekt-Besprechung vom 11.12.2021 (Abbildung \ref{fig:2021-12-11-Projekt-Besprechung}):

\begin{itemize}
    \item Eintrag ins Gamelog zum Spielstart (Commit \url{https://git.io/JyBRf}).
    \item Rundensystem implementieren. Dazu mindestens notwendig: Lebens- und Angriffspunkte der User-Chars sowie des Gegners. Ablauf einer Runde als Pseudo-Code: 
    \begin{enumerate}
        \item Gameloop-Schleife: [round-state]
        \item Aktion von Gegner ausführen (Schaden) [100]
        \item Prüfen ob User-Char tod ist (HP < 1 = Dead-Flag: True) [200]
        \item Prüfen wie viele User-Chars noch leben (n < 1 = Gameover-Flag: True, break-Gameloop-Schleife) [300]
        \item Aktionen der User-Chars aufnehmen (Entscheidung für nächste Aktion von jedem Spieler annehmen + wegspeichern) [400]
        \item Alle Aktionen der User-Chars ausführen (Aktionen laden und ausführen: Schaden, Aktion, Passen) [500]
        \item Nach jedem Spieler, prüfen ob Gegner besiegt wurde (HP < 1 = Win-Flag: True, break-Gameloop-Schleife) [immernoch 500]
        \item Rundencounter +1 [600]
        \item Gameloop-Schleife nächster Durchlauf [700, zurück zu 100]...
    \end{enumerate}
    Steuerung über "round-stat" Hilfsvariable, gespeichert in Games-Tabelle (Default=0, Gameover=990, Win=995). Da der Aufruf der Spiele-Logik über den Websocket-Heartbeat der Spieler erfolgt, müssen die Arbeitsschritte sehr kleinteilig sein und diese laufend in kleinen (kleinsten?) Schritten weggepeichert werden. Möglicherweise ergibt sich ein Sync-Problem.
\end{itemize}



	% 




\section{Aufgabenbeschreibung}
Beispielinhalt und Texte:
Das Thema der Seminararbeit gründet als Idee vor allem aus den folgenden beiden Beobachtungen aus privatem Lebensalltag und beruflicher Praxis: 
\begin{enumerate}
	\item Die seit den 2010er Jahren aufkommenden \textbf{smarten Technologien} finden im öffentlichen Raum und im Lebensalltag breiter Bevölkerungsschichten immer mehr Einzug. Drei Anwendungsbeispiele verdeutlichen und belegen das: 
	\begin{itemize}
	 \item Smart Watch: Uhren die z.B. um Fitness- und Nachrichtenfunktionen erweitert sind.
	 \item Smart TV: Fernsehgeräte die z.B. Zugriff auf Internet-Mediatheken erlauben.
	 \item Smart Home: Hausautomatisierungen und -steuerungen im privaten Bereich für  z.B. Licht und Heizung.
	 \end{itemize}

	 \item Das Aufkommen und der Erfolg der \textbf{PropTech\footnote{Kofferwort, englisch, aus 'property' (Immobilien) und 'technology' (Technologiegen)}-Branche} sowie die dieser Branche zuzuordnenden Unternehmen belegt die Relevanz von Innovation für die Wohnungswirtschaft ebenso wie die folgenden Erkenntnisse einer Studie \parencite[S. 18]{zia}:
	 \begin{itemize}
		 \item 72\% der befragten immobilienwirtschaftlichen Unternehmen nehmen Effizienzsteigerungen in Kernprozessen durch Einsatz digitaler Technologien an.
		 \item Weiter: Über ein Drittel gehen davon aus, dass Neugeschäft durch Einsatz digitaler Technologien generiert werden kann. 
	 \end{itemize}
\end{enumerate}

Vorgenannte Beobachtungen führen zu folgender Hypothese:

 \Rightarrow \textbf{Der Einsatz von smarten Displays in Quartieren der \ac{WoWi} ist möglich, wirtschaftlich und innovativ.}



\subsection{Was ist das Ziel der Projektarbeit?}
Beispielinhalt und Texte:
Als mögliche Standorte ergeben sich in Anlehnung an bisher übliche \glqq{}schwarze Bretter\grqq{} und \glqq{}Schaukästen\grqq{} in den Gebäuden sowie auch im Außenbereich vorhandenen Anlagen (Schaukästen, Werbeanlagen) auch für smarte Displays einige Vor- und Nachteile (Pro und Contra) die wie folgt stichpunktartig beschrieben werden: 

 \begin{itemize}
	 \item \textbf{Innenbereich:} 
	 \\ Wandmontage ebenso wie üblich und bekannte Schaukästen und schwarze Bretter im Windfang oder dem Etagenflur im Erdgeschoss eines Mietshauses.\\
	 \textbf{Pro:}\\
	  - Weiternutzung vorhandener, bewährter Montageorte \\
	  - Baurechtlich genehmigungsfrei \\
	  - Anbindung an Strom und Internet leichter als im Außenbereich \\
	  - Günstigere Bauart (kann weniger witterungsfest und vandalismussicher sein)\\
	 \textbf{Contra:}\\
	  - Wird wenig innovative Bauart zur Folge haben\\
	  - Kleinere Anzeigeflächen \\
	  - Nutzerkreis umfasst nur Mieter und Besucher des Hauses
	 \item \textbf{Außenbereich:} 
	 \\ Als freistehende Installation vor einem Wohngebäude, an einem markanten Wegepunkt im Quartier oder einem Innenhof eines Gebäudekomplexes.\\
	 \textbf{Pro:}\\
	  - Höhere Sichtbarkeit und Reichweite (nicht nur Mieter und Besucher eines Hauses, sondern auch Umfeld, Nachbarn und Durchgangsverkehr)  \\
	  - Attraktive Bauarten möglich \\
	  - Eröffnet weitergehende Nutzungsmöglichkeiten \\
	 \textbf{Contra:}\\
	  - Höhere Investitionskosten durch dem Standort geschuldete Bauart und Größe \\
	  - Erdarbeiten werden in der Regel nötig sein (Stromanschluss)  \\
	  - Baurechtlich genehmigungspflichtig, verursacht Mehrkosten und Aufwand
	  \item \textbf{Übergangsbereich Hauseingangstüre:} 
	  \\ Anbringung an Flächen die das Gebäude und den Außenbereich verbinden, beispielhaft genannt hier: Seitenteil der Türelemente im Hauseingangsbereich.\\
	  \textbf{Pro:}\\
	   - Unter Umständen umsetzbar als Modernisierungsmaßnahme (Nutzung als Videogegensprechanlage)  \\
	   - Technikaverse Bewohner und Besucher kommen zwangläufig in Kontakt mit dem smart Display \\
	  \textbf{Contra:}\\
	   -  Vorteile ergeben sich z.T. nur in noch nicht modernisiertem Bestand \\
	   -  Kleinerer Nutzerkreis und Anzeigefläche \\
 \end{itemize}

Eine weitergehende Bewertung oder Befürwortung einzelner Standorte soll hier nicht erfolgen um einzelne Anwendungsfälle nicht hier schon auszuschließen.


\subsection{Worin bestehen die (wahrscheinlichen) Herausforderungen? (allg. technisch und auch persönlich)}
Beispielinhalt und Texte:
Der Begriff soll daher hier geschärft werden um in der weiteren Verwendung un­miss­ver­ständ­lich zu sein.  

Die Definition erfolgt unter Bezug auf
\begin{enumerate}
	\item  die erfolgte Abgrenzung in 'enge' und 'weite' Definition nach \citeauthor{wirtschaftsfaktorimmo} (\citeyear[S. 9]{wirtschaftsfaktorimmo}) sowie
	\item die institutionelle Systematisierung der Immobilienwirtschaft nach \citeauthor{brauer2011einfuhrung} (\citeyear[S. 26]{brauer2011einfuhrung}).
\end{enumerate}

Die daraus entnommenen Zitate sollen hier im weiteren als gültige Definition für den verwendeten Begriff \textbf{Wohnungswirtschaft (WoWi)} gelten:  
\begin{itemize}
	\item Aus \citeauthor{wirtschaftsfaktorimmo} (\citeyear[S. 9]{wirtschaftsfaktorimmo}): \glqq{}alle Unternehmen, die an der Bewirtschaftung, Vermittlung und Verwaltung von Immobilien unmittelbar beteiligt sind\grqq{}. 
	 \item Sowie nach \citeauthor{brauer2011einfuhrung} (\citeyear[S. 36]{brauer2011einfuhrung}) zu den unterschiedlichen Rechtsformen \glqq{}(...)kommunalen, genossenschaftlichen und privaten Unternehmen(...)\grqq{} und den gleichwohl identischen Aufgabenfeldern: \glqq{}(...)nachhaltige Vermietung und Bestandsmanagement(...)\grqq{}.
\end{itemize}





\newpage
\section{Anforderungen}
Beispielinhalt und Texte:
Die genaue Zuordnung stellt sich also wie folgt dar:

\begin{table}[H]
	\caption{Zuordnung der Anforderungen der Hypothese zu den kritischen Anforderungen des Proof of Concept (PoC)}
	\label{tbl:zuordnungHypothesePoc}
	\begin{tabularx}{\textwidth}[ht]{|l|c|X|}
	\hline
	\textbf{Anforderung der Hypothese} & \textbf{Zuordnung} & \textbf{Abbildung in \ac{PoC}} \\
	\hline\hline 
	ist möglich & \Leftrightarrow & Prüfung der Machbarkeit  \\
	\hline 
	ist wirtschaftlich & \Leftrightarrow &  Effizienz-Faktoren aufzeigen \\
	\hline 
	ist innovativ & \Leftrightarrow &  Nutzbarkeit und Anwendungsfälle \\
	\hline
	\end{tabularx}
\end{table}


\subsection{Welche Techniken/ Technologien sollen eingesetzt werden, um die Aufgabe zu lösen/ realisieren?}
Beispielinhalt und Texte:

Eine Betrachtung der Wirtschaftlichkeit einer Einzelinvestition in ein smart Display in einem Quartier könnte damit nach z.B. folgendem Schema erfolgen:


\begin{table}[H]
	\caption{Mögliches Schema einer Wirtschaftlichkeits- und Effizienzbetrachtung der Einzelinvestition in smart Displays}
	\label{tbl:SchemaWBsmartDisplay}
	\begin{tabularx}{\textwidth}[ht]{cl}
	\hline
	\textbf{ - }   &  Investitionskosten (der Einzelmaßnahme) \\
	\textbf{ - }   &  lfd. Betriebs- und Wartungskosten  \\
	\textbf{ + }   &  lfd. Einsparung Personalkosten  \\
	\textbf{ + }   &  Refinanzierung als Modernisierungsmaßnahme   \\
	\textbf{ + }   &  Umlage von (Teilen der) Betriebskosten auf Gebäudenutzer  \\
	\textbf{ + }   &  Mehrerlöse durch Überlassung als Werbefläche an Dritte  \\
	\hline\hline
	\textbf{ = }   &  \textbf{in Euro messbare Wirtschaftlichkeit} \\
	\textbf{ + }   &  Digitalisierung eines Geschäftsprozesses  \\
	\textbf{ + }   &  Imagegewinn für das Quartier \\
	\textbf{ + }   &  Öffentlichkeitswirksame Einführung und Realisierung  \\
	\hline\hline
	\textbf{ = }   &  \textbf{gesamt zu bewertende Effizienz} \\
	\hline
\end{tabularx}
\end{table}

Kommt man nun zurück auf die Definition des Bergriffs der Effizienz nach \citeauthor{eichhorn2016} (\citeyear[S. 183 f.]{eichhorn2016}), kann man feststellen, dass es für die Beurteilung wesentliche Faktoren gibt, die nicht der Wirtschaftlichkeit zuzurechnen sind.



\subsection{Warum sollen gerade diese eingesetzt werden?}
Beispielinhalt und Texte:
Mit smarten Displays sind in dieser Seminararbeit nicht die von Desktop-Computer abnehmbaren und tragbaren LCD-Monitore gemeint die 2002 im Zusammenhang mit Microsofts Betriebssystem \glqq{}Windows CE for Smart Displays\grqq{} vorgestellt wurden \parencite{heise-ms-sd} und im Jahr 2004 wieder eingestellt worden sind \parencite{ct-3-2004}. 

\begin{figure}[H]
	\caption{Handteil eines smart Displays nach Microsoft-Konzept}\label{fig:HandteilMSsmartDisplay}
	\includegraphics[height=5cm,keepaspectratio]{2021-11-29-Entwurf-Klassen-Ui}
	\\
	Quelle: Homepage von Mark Strehlow, Senior Interaction Designer\\ im Projekt Mira (\href{https://msdo.us/Microsoft-Mira}{msdo.us/Microsoft-Mira})
\end{figure}

Vielmehr sind hier erst noch durch Forschung und Entwicklung für die \ac{WoWi} zu schaffenden und nutzbar zu machenden Geräte gemeint.


\subsection{Gibt es besondere Anforderungen? (technisch, Benutzer, sonstige)}
Beispielinhalt und Texte:

\newpage
\section{Herangehensweise}
Beispielinhalt und Texte:

\subsection{Wie soll das Ziel erreicht werden (Vorgehen, Architektur)}
Beispielinhalt und Texte:

\newpage
\section{Vorstellung des Ergebnisses}
Beispielinhalt und Texte:

\newpage
\section{Reflektion}
Beispielinhalt und Texte:

 die Anwendung wird hier durch eine Betrachtung  der untenstehenden Punkte erfolgen: 
\begin{itemize}
	\item \textbf{Prüfung rechtlicher und IT-technischer Machbarkeit:}
	\\ Neben baurechtlicher Betrachtung hier auch Prüfung in Bezug auf verschiedenartige Realisierungen in Größe, Standort, Bauart und IT-Integration.
	\item \textbf{Aufzeigen und erörtern relevanter Effizienz-Faktoren:}
	\\ Umfasst typisch erwartbare Effizienz-Faktoren wie die Wirtschaftlichkeit ebenso wie auch darüber hinausgehende Auswirkungen die zur ebenso zur Effizienz zu zählen sind.
	\item \textbf{Darstellung möglicher Nutzbarkeiten und Anwendungsfälle:}
	\\ Ausführungen zu erwartbaren Einflüssen auf vorhandene Geschäftsprozesse in der \ac{WoWi} aber auch durch Aufzeigen von neuartigen und darüber hinausgehenden Anwendungsfällen und -bereichen.
\end{itemize}

Diese drei vorgenannten Punkte spiegeln die drei Anforderungen aus der Hypothese der Einleitung wieder und entsprechen dieser genau, in den jeweils genannten Reihenfolgen. 




\end{appendices}
\addtocontents{toc}{\protect\setcounter{tocdepth}{2}}






\newpage % Literaturverzeichnis

% Im Literaturverzeichnis "und" wieder durch "&" ersetzen	
	\DeclareDelimFormat*{finalnamedelim}{\addspace\&\space}

% Punkt hinter und vor der Jahreszahl entfernen	- Wichtig für Quell-Arten wie misc und online -- Sonst ein überflüssiger Punkt im LitVerz.
	\renewbibmacro*{author}{%
	\printtext{%
	\ifnameundef{author}
	{\usebibmacro{labeltitle}}
	{\printnames[apaauthor][-\value{listtotal}]{author}%
	\setunit*{\addspace}%
	\printfield{nameaddon}%
	\ifnameundef{with}
	{}
	{\setunit{}\addspace\mkbibparens{\printtext{\bibstring{with}\addspace}%
	\printnames[apaauthor][-\value{listtotal}]{with}}
	\setunit*{\addspace}}}%
	% \newunit\newblock%
	\usebibmacro{labelyear+extradate}}}

\printbibliography[heading=bibintoc,title=Literaturverzeichnis]

\newpage % Ehrenwörtliche Erklärung
\pagenumbering{gobble} % Keine Seitenzahlen mehr
\section*{Ehrenwörtliche Erklärung} 
	Hiermit versichere ich, dass die vorliegende Arbeit von mir selbstständig und ohne unerlaubte Hilfe angefertigt worden ist, insbesondere dass ich alle Stellen, die wörtlich oder annähernd wörtlich aus Veröffentlichungen entnommen sind, durch Zitate als solche gekennzeichnet habe. Weiterhin erkläre ich, dass die Arbeit in gleicher oder ähnlicher Form noch keiner Prüfungsbehörde/Prüfungsstelle vorgelegen hat. Ich erkläre mich damit \textbf{nicht einverstanden}, dass die Arbeit der Öffentlichkeit zugänglich gemacht wird. Ich erkläre mich damit einverstanden, dass die Digitalversion dieser Arbeit zwecks Plagiatsprüfung auf die Server externer Anbieter hochgeladen werden darf. Die Plagiatsprüfung stellt keine Zurverfügungstellung für die Öffentlichkeit dar.

			\par\medskip
			\par\medskip

			\vspace{5cm}

			\begin{table}[H]
				\centering
				\begin{tabular*}{\textwidth}{c @{\extracolsep{\fill}} ccccc}
					
					\myOrt, \the\day.\the\month.\the\year
					&
					\includegraphics[width=0.35\textwidth]{unterschrift_rico}\vspace*{-0.35cm}
					\\
					\rule[0.5ex]{12em}{0.55pt} & \rule[0.5ex]{12em}{0.55pt} \\
					(Ort, Datum) & (Rico ) 
					\\

					
					  
					&
					\includegraphics[width=0.35\textwidth]{unterschrift_henning}\vspace*{-0.35cm}
					\\
					 & \rule[0.5ex]{12em}{0.55pt} \\
					& (Henning ) 
					\\

					
					  
					&
					\includegraphics[width=0.35\textwidth]{unterschrift_julian.png}\vspace*{-0.35cm}
					\\
					 & \rule[0.5ex]{12em}{0.55pt} \\
					 & (Julian ) 
					\\



				\end{tabular*} \\
			\end{table}

\end{document}