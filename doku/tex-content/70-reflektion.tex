
\section{Reflektion}

An dieser Stelle finden sich stichpunktartig einige Reflektionen von uns, die sich zum Abschluss der Arbeit ergaben und aus unserer Sicht hier durchaus nennenswert sind. 

    \textbf{Rundenbasierter vs. chaotischer Spielablauf:} Der Ansatz die Entwicklung durch den Einsatz eines rundenbasierten Ablaufs bzw. Kampfsystems deutlich zu vereinfachen, musste im Laufe der Entwicklung zumindest angezweifelt werden. Denn der Aufwand, der durch die notwendige Konzeptionierung und Detailplanung entsteht, ist nicht zu unterschätzen. Dementgegen stünde bei einem chaotischen Spielablauf lediglich das Handling der Events. 

    \textbf{Kleinteilige Aufgabenpakete:} In der Entwicklung eine überschaubare Anzahl an kleinteiligen bzw. Teilaufgaben vor sich zu haben, empfanden wir als sehr hilfreich. Man hat damit einen Überblick über die Arbeit der nächsten Tage. Bei der Entwicklung der Rundenlogik entstand durch die Kenntniss um die nächsten Rundenschritte hier ein stets guter Überblick. Der Abschluss jeder einzelnen Teilaufgabe sorgte weiter laufend für positive Motivation.

    \textbf{Lernkurve und Codequalität:} Eigentlich müsste am Ende eines Entwicklungsprojektes stets noch einmal von vorne anfangen. Allein um alles zu korrigieren und anzupassen bzw. auf einen gleichen Quaitätsstand zu bringen, was im Projektverlauf bei den Beteiligten an Fähigkeiten und Wissen gelernt wurde.

    \textbf{Setting:} Rückwirkend betrachtet, scheint die Wahl des Settings, abhängig von Spielmechanismen, die nachher nicht umgesetzt wurden, nicht mehr so entscheidend, wie zu dem Zeitpunkt der Entwicklung. Denn so wie das Spiel jetzt aufgebaut ist, hätte es durchaus auch als Science Fiction funktionieren können. Es müssten lediglich der optische Auftritt und die Geschichten abgeändert werden. Der technische Unterbau, speziell der des Kampfmechanismus, welcher mitentscheidend für die Änderung war, ist zum gegenwertigen Zeitpunkt sehr universell und einfach gehalten und könnte so ohne Änderungen übernommen werden.

    \textbf{Storytelling/Szenenentwicklung:} Damit ist das Team im Großen und Ganzen sehr zufrieden. Die Zusammenarbeit zwischen Autor und Artist war engmaschig und hat gut funktioniert. Es war jedoch nicht immer einfach, nicht zu weit abzuschweifen oder die Szene zu komplex werden zu lassen, denn der Autor hätte sich manchmal noch mehr Tiefe gewünscht, um all seine Ideen umzusetzen. Außerdem merkt der Spieler im fertigen Spiel nicht, welcher Aufwand betrieben wurde, um die Texte und die Szenen zu entwickeln. Für das anvisierte Ziel, welches zu Anfang des Projektes gesteckt wurde, nämlich technische Umsetzbarkeit und Implementierung aller Anforderungen, war der Umfang vielleicht jetzt schon etwas viel. Denn der Aufwand, im Besonderen im Grafikdesign, war schon recht hoch.

    \textbf{Balancing:} Das Balancing funktioniert grundsätzlich gut. Die Tatsache, dass der Charakter seine Erfahrung getrennt für HP und AP ausgeben und dadurch eine extreme Schere zwischen den Charakteren entstehen kann, macht das Balancing aber auch recht schwer. Hier wäre es vielleicht mit einem Ansatz, mit festen Erfahrungspunkten je Gegner und ein starres Levelsystem, bei dem man bei festgelegten Grenzen einen Level aufsteigt und feste HP und AP Zuwächse je Level bekommt, einfacher gewesen. Dadurch ließen sich die Gegner in den aufeinander folgenden Leveln viel besser skalieren, gleichzeitig würde aber die Variabilität und Individualität der Charaktere stark beschnitten werden.
    
    \textbf{Grafiken \& Animationen:} Der Arbeitsaufwand für die Erstellung der Intro- und Hintergrundgrafiken und die Animation der Gegner, wurde von uns zu Beginn der Entwicklung stark unterschätzt. Selbst ohne die Zeit für die Bewältigung von einhergehenden Problemen, wie z.B. dem Erlernen halbwegs vorzeigbarer Zeichentechniken, mitzuzählen, waren viele Stunden notwendig, um alle Assets fertigzustellen. Auf der anderen Seite machen aber gerade Grafikdesign und -stil einen sehr großen Teil des Charmes und der Identität eines Spieles aus. Es ist nun aber besser nachzuvollziehen, warum sich bei der Entwicklung von Spielen durch professionelle Studios ganze Abteilungen um nichts anderes kümmern als um Concept Art, Animation und Grafik-Design.
    
    \textbf{Verwendung eines CSS-Frameworks:} Zu Anfang der Entwicklung hatten wir angedacht benötigte Assets für die Frontend-Entwicklung vollständig selbst zu erstellen. So war auch die grafische Aufbereitung von HTML-Elementen zunächst etwas, was wir selbst machen wollten. Hier wurde jedoch schnell klar, dass dies einen enormen Aufwand notwendig machen würde und den Rahmen des verhältnismäßig kleinen Projektes sprengen würde. Das hervorragende CSS-Framework RPGUI konnte an dieser Stelle sehr gewinnbringend eingesetzt werden und hat uns beim Frontend-Design sehr geholfen.
    
    \textbf{Genereller Lernkurve:} Vor diesem Projekt verfügten die beteiligten Studenten über keine oder nur sehr geringe Erfahrung im Bereich der Webentwicklung. Weder die Abläufe einer Spielentwicklung noch das Django-Framework waren uns näher bekannt. Da am Ende ein einfaches, aber funktionierendes Spiel dabei herausgekommen ist, kann man behaupten, dass die Lernkurve aller Projektmitarbeiter entsprechend steil gewesen ist. Wir werden sicherlich einiges an Erfahrungen und Best Practices aus diesem Projekt für die Zukunft mitnehmen können.   


