


\section{Projektnotizen}

Austausch und Zusammenarbeit erfolgte auf verschiedenen Platformen:
\begin{itemize}
    \item Gezeichnet und Entwürfe wurden meist in Miro\footnote{Siehe \url{https://miro.com/app/board/uXjVOdN2haQ=/}} erstellt.
    \item Besprechungen erfolgten meist in Teams\footnote{Siehe \url{https://teams.microsoft.com/l/team/19\%3aDoBvOwOIC6WNhsL9kOIYFKNtVftU1yBtcEn_gcyQtcg1\%40thread.tacv2/conversations?groupId=850a22ff-34a2-4fe2-a506-f55ac4d595f8&tenantId=b9b6f99a-a243-422d-ab36-f726574c981a}}. 
    \item Der gemeinsame Code und die Dokumentation wurden auf Github erstelt: \url{https://github.com/tstsrv-de/rpg}. 
\end{itemize}




\section{Ideen für Erweiterungen über diese Projektarbeit hinaus (Ausblick)} \label{ausblick}

Da der Umfang dieser Projekt beschränkt ist, können nicht alle erdachten Funktionen umgesetzt werden. Einige Ideen, sollen aber nicht ungenannt bleiben: 
\begin{enumerate}
    \item Würfel bei Schaden bzw. Angriff implementieren.
    \item Aggrotabelle und verbundene Funktionen implementieren.
    \item Inaktive und getrennte Spieler bzw. Nutzer aus den Chatfunktionen der Weltkarte und Lobby entfernen. Ggfls. auch einen Timer für das Spiel anlegen der anderen Spielern anzeigt, wenn ein Spieler inaktiv bzw. getrennt vom Server ist. 
    \item E-Mail Funktion von Django konfigurien, damit das Zurücksetzen von Passwörtern u.A. möglich wird.
\end{enumerate}


\section{Inhalte aus und zu Projektbesprechungen}

Stets Sonntags erfolgten Projektbesprechungen. Notizen und Zusammenfassungen davon finden sich hier in fortschreitender, chronologischer Reihenfolge (insofern diese im Hauptteil der Dokumentation nicht bereits dargetellt wurden). Ebenso hier entsprechend einsortiert, finden sich Konzeptzeichnungen und Entwürfe aller Art (UI, Code, Datenbankmodelle).


(TODO!) Bildbeschreibungen ergänzen, wichtige Bilder beschreiben.


\begin{figure}[H]
    \centering
    \caption{2021-11-23-erster-entwurf-gameloop}
    \label{fig:2021-11-23-erster-entwurf-gameloop}
    \includegraphics[width=1\textwidth]{2021-11-23-erster-entwurf-gameloop}
\end{figure}


\begin{figure}[H]
    \centering
    \caption{2021-11-23-erstes-db-konzept}
    \label{fig:2021-11-23-erstes-db-konzept}
    \includegraphics[width=1\textwidth]{2021-11-23-erstes-db-konzept}
\end{figure}


\begin{figure}[H]
    \centering
    \caption{2021-11-27-erstentwurf-ui}
    \label{fig:2021-11-27-erstentwurf-ui}
    \includegraphics[width=1\textwidth]{2021-11-27-erstentwurf-ui}
\end{figure}


\begin{figure}[H]
    \centering
    \caption{2021-11-29-Entwurf-Klassen-Ui}
    \label{fig:2021-11-29-Entwurf-Klassen-Ui}
    \includegraphics[width=1\textwidth]{2021-11-29-Entwurf-Klassen-Ui}
\end{figure}


\begin{figure}[H]
    \centering
    \caption{2021-11-30-Entwurf-Lobby-Logik}
    \label{fig:2021-11-30-Entwurf-Lobby-Logik}
    \includegraphics[width=1\textwidth]{2021-11-30-Entwurf-Lobby-Logik}
\end{figure}


\begin{figure}[H]
    \centering
    \caption{2021-11-30-Entwurf-Lobby-UI}
    \label{fig:2021-11-30-Entwurf-Lobby-UI}
    \includegraphics[width=1\textwidth]{2021-11-30-Entwurf-Lobby-UI}
\end{figure}


\begin{figure}[H]
    \centering
    \caption{2021-12-02-Countdown-Logik}
    \label{fig:2021-12-02-Countdown-Logik}
    \includegraphics[width=1\textwidth]{2021-12-02-Countdown-Logik}
\end{figure}

 
\begin{figure}[H]
    \centering
    \caption{2021-12-05-Projketbesprechung-Miro-b}
    \label{fig:2021-12-05-Projketbesprechung-Miro-b}
    \includegraphics[width=1\textwidth]{2021-12-05-Projketbesprechung-Miro-b}
\end{figure}


\begin{figure}[H]
    \centering
    \caption{2021-12-05-Projketbesprechung-Miro-c}
    \label{fig:2021-12-05-Projketbesprechung-Miro-c}
    \includegraphics[width=1\textwidth]{2021-12-05-Projketbesprechung-Miro-c}
\end{figure}


\begin{figure}[H]
    \centering
    \caption{2021-12-05-Projketbesprechung-Miro-d}
    \label{fig:2021-12-05-Projketbesprechung-Miro-d}
    \includegraphics[width=1\textwidth]{2021-12-05-Projketbesprechung-Miro-d}
\end{figure}


\begin{figure}[H]
    \centering
    \caption{2021-12-11-Projekt-Besprechung-Klassenbeschreibung}
    \label{fig:2021-12-11-Projekt-Besprechung-Klassenbeschreibung}
    \includegraphics[width=1\textwidth]{2021-12-11-Projekt-Besprechung-Klassenbeschreibung}
\end{figure}


\begin{figure}[H]
    \centering
    \caption{11.12.2021: Projekt Besprechung: Gegenseitiges Update und Wechsel von Szenenlogik zu Kampfsystem für das RPG }
    \label{fig:2021-12-11-Projekt-Besprechung}
    \includegraphics[width=1\textwidth]{2021-12-11-Projekt-Besprechung}
\end{figure}


\begin{figure}[H]
    \centering
    \caption{06.01.2022: Projekt Besprechung: Ermitteln von restlichen ToDos, Aufgabenverteilung, Meilienstein- und Terminplanung}
    \label{fig:2022-01-06-Projektbesprechung}
    \includegraphics[width=1\textwidth]{2022-01-06-Projektbesprechung}
\end{figure}


\begin{figure}[H]
    \centering
    \caption{08.01.2022: Diagram: Seitenaufbau und Spielablauf}
    \label{fig:2022-01-08-Spielablauf-Chart}
    \includegraphics[width=1\textwidth]{2022-01-08-Spielablauf-Chart}
\end{figure}

