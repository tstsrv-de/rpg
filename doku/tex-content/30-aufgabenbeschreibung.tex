

\section{Aufgabenbeschreibung}
Am Ende der Spieleentwicklung soll als primäres Ziel der Projektarbeit ein RPG (Roleplaying Game) mit Fantasy-Setting entstehen, bei dem es vor allen Dingen um die Fertigstellung des kompletten Spielumfangs und nicht nur einiger Teilbereiche geht. Das Spiel soll also von Anfang bis Ende durchgespielt werden können, alle an das Projekt gestellten Anforderungen erfüllen und die gewünschten Inhalte bieten. Im Einzelnen sind im o.g. Zusammenhang folgende Aspekte zu nennen. Das Spiel muss mit einem gängigen Browser online spielbar sein, es soll sowohl Einzelspieler- als auch Mehrspielerelemente enthalten und es soll verschiedene Charakterklassen geben, die sich in ihren Fähigkeiten klar voneinander unterscheiden. Im Laufe des Spiels sollen sich, bei Erreichen bestimmter Grenzen in einem für jeden Charakter eigenem Erfahrungspools, diese Fähigkeiten verbessen. Des Weiteren wird es vom Projektteam unabhängig vom Endproduckt ebenfalls als primäres Ziel angesehen, den Projektablauf erfolgreich zu gestalten.

Als sekundäre Ziele, also den primären Zielen klar untergeordnet, sieht die Projektgruppe die Spielerfahrung der einzelnen Spieler. Es ist allerdings angedacht diese in den Rahmen der Möglichkeiten so intensiv wie möglich zu gestalten, jedoch steht wie oben bereits erwähnt die Umsetzung der Vorgaben im Vordergrund. Außerdem möchte jeder der Teilnehmer an diesem Projekt Erfahrung im Spieldesign sammeln.

Die Herausforderungen in diesem Projekt sind recht vielfältig und sind grob in technische und persönliche zu gliedern.

Zu Ersterem ist zu zählen, dass zu Beginn des Projekts nicht klar ist, ob das vom Team erst einmal theoretisch entwickelte Konzept auch real technisch umzusetzen ist. Zum Einen, weil nicht sicher ist, ob die gewählte technische Plattform geeignet ist und zum Anderen, weil nicht klar ist ob das Know-How des Teams reicht, um das Gewollte in ein fertiges Endprodukt zu gießen. Außerdem könnten Rechte an Grafiken und vielleicht sogar Programmcode zum Problem werden.

Die persönlichen Herausforderungen liegen darin, das ganze Projekt zeitlich so zu gestalten, dass es neben der hauptberuflichen Tätigkeit des Projektteams noch genug Zeit für die Umsetzung bleibt. Krankheit oder Quarantäne in der aktuellen Pandemie von einzelnen Teammitgliedern könnten den zeitlichen Rahmen ebenfalls in Gefahr bringen. Auch muss für jedes Mitglied des Teams ein geeignetes Aufgabenfeld gefunden werden, so dass die individuellen Fähigkeiten sich möglichst ideal ergänzen, um z.B. eine Doppelbelastung eines der Teammitglieder zu vermeiden und die zur Verfügung stehende Zeit optimal zu nutzen.




