


In diesem Abschnitt sind aktuell einige Unterlagen eingefügt, die im Rahmen des Projektes eine Relevanz hatten. Vor Abgabe der Projektarbeit, soll dieser Abschnitt überarbeitet, ausgedünnt und ergänzt werden. 

\section{Projektnotizen}

Austausch und Zusammenarbeite erfolgte auf verschiedenen Platformen:
\begin{itemize}
    \item Gezeichnet und Entwürfe wurden meist in Miro erstellt: \url{https://miro.com/app/board/uXjVOdN2haQ=/}. 
    \item Besprechungen erfolgten meist in Teams: \url{https://teams.microsoft.com/l/team/19%3aDoBvOwOIC6WNhsL9kOIYFKNtVftU1yBtcEn_gcyQtcg1%40thread.tacv2/conversations?groupId=850a22ff-34a2-4fe2-a506-f55ac4d595f8&tenantId=b9b6f99a-a243-422d-ab36-f726574c981a}. 
    \item Der gemeinsame Code und die Dokumentation wurden auf Github erstelt: \url{https://github.com/tstsrv-de/tstsrv-de}. 
\end{itemize}

\subsection{Projektbesprechungen}

Stets Sonntags erfolgten Projektbesprechungen. Notizen und Zusammenfassungen davon finden sich hier in fortschreitender, chronologischer Reihenfolge. Ebenso hier entsprechend einsortiert, finden sich Konzeptzeichnungen und Entwürfe aller Art (UI, Code, Datenbankmodelle).

(TODO!) Bildbeschreibungen ergänzen, wichtige Bilder beschreiben.

2021-11-23-erster-entwurf-gameloop 
\begin{figure}[H]
    \centering
    \caption[]{2021-11-23-erster-entwurf-gameloop}
    \label{fig:2021-11-23-erster-entwurf-gameloop}
    \includegraphics[width=1\textwidth]{2021-11-23-erster-entwurf-gameloop}
\end{figure}

2021-11-23-erstes-db-konzept 
\begin{figure}[H]
    \centering
    \caption[]{2021-11-23-erstes-db-konzept}
    \label{fig:2021-11-23-erstes-db-konzept}
    \includegraphics[width=1\textwidth]{2021-11-23-erstes-db-konzept}
\end{figure}

2021-11-27-erstentwurf-ui 
\begin{figure}[H]
    \centering
    \caption[]{2021-11-27-erstentwurf-ui}
    \label{fig:2021-11-27-erstentwurf-ui}
    \includegraphics[width=1\textwidth]{2021-11-27-erstentwurf-ui}
\end{figure}

2021-11-27-projektskizze-1 
\begin{figure}[H]
    \centering
    \caption[]{2021-11-27-projektskizze-1}
    \label{fig:2021-11-27-projektskizze-1}
    \includegraphics[width=1\textwidth]{2021-11-27-projektskizze-1}
\end{figure}

2021-11-27-projektskizze-2 
\begin{figure}[H]
    \centering
    \caption[]{2021-11-27-projektskizze-2}
    \label{fig:2021-11-27-projektskizze-2}
    \includegraphics[width=1\textwidth]{2021-11-27-projektskizze-2}
\end{figure}

2021-11-29-Entwurf-Klassen-Ui 
\begin{figure}[H]
    \centering
    \caption[]{2021-11-29-Entwurf-Klassen-Ui}
    \label{fig:2021-11-29-Entwurf-Klassen-Ui}
    \includegraphics[width=1\textwidth]{2021-11-29-Entwurf-Klassen-Ui}
\end{figure}

2021-11-30-Entwurf-Lobby-Logik 
\begin{figure}[H]
    \centering
    \caption[]{2021-11-30-Entwurf-Lobby-Logik}
    \label{fig:2021-11-30-Entwurf-Lobby-Logik}
    \includegraphics[width=1\textwidth]{2021-11-30-Entwurf-Lobby-Logik}
\end{figure}

2021-11-30-Entwurf-Lobby-UI 
\begin{figure}[H]
    \centering
    \caption[]{2021-11-30-Entwurf-Lobby-UI}
    \label{fig:2021-11-30-Entwurf-Lobby-UI}
    \includegraphics[width=1\textwidth]{2021-11-30-Entwurf-Lobby-UI}
\end{figure}

2021-12-02-Countdown-Logik 
\begin{figure}[H]
    \centering
    \caption[]{2021-12-02-Countdown-Logik}
    \label{fig:2021-12-02-Countdown-Logik}
    \includegraphics[width=1\textwidth]{2021-12-02-Countdown-Logik}
\end{figure}

2021-12-05-Projketbesprechung-Miro-b 
\begin{figure}[H]
    \centering
    \caption[]{2021-12-05-Projketbesprechung-Miro-b}
    \label{fig:2021-12-05-Projketbesprechung-Miro-b}
    \includegraphics[width=1\textwidth]{2021-12-05-Projketbesprechung-Miro-b}
\end{figure}

2021-12-05-Projketbesprechung-Miro-c 
\begin{figure}[H]
    \centering
    \caption[]{2021-12-05-Projketbesprechung-Miro-c}
    \label{fig:2021-12-05-Projketbesprechung-Miro-c}
    \includegraphics[width=1\textwidth]{2021-12-05-Projketbesprechung-Miro-c}
\end{figure}

2021-12-05-Projketbesprechung-Miro-d 
\begin{figure}[H]
    \centering
    \caption[]{2021-12-05-Projketbesprechung-Miro-d}
    \label{fig:2021-12-05-Projketbesprechung-Miro-d}
    \includegraphics[width=1\textwidth]{2021-12-05-Projketbesprechung-Miro-d}
\end{figure}

2021-12-11-Projekt-Besprechung-Klassenbeschreibung 
\begin{figure}[H]
    \centering
    \caption[]{2021-12-11-Projekt-Besprechung-Klassenbeschreibung}
    \label{fig:2021-12-11-Projekt-Besprechung-Klassenbeschreibung}
    \includegraphics[width=1\textwidth]{2021-12-11-Projekt-Besprechung-Klassenbeschreibung}
\end{figure}


\begin{figure}[H]
    \centering
    \caption[]{11.12.2021: Projekt Besprechung: Gegenseitiges Update und Wechsel von Szenenlogik zu Kampfsystem für das RPG }
    \label{fig:2021-12-11-Projekt-Besprechung}
    \includegraphics[width=1\textwidth]{2021-12-11-Projekt-Besprechung}
\end{figure}



\section{Entwicklungsnotizen}

\subsection{Entwicklung, Sonntag 19.12.2021:}

An diesem Tag wurden weitere, notwenige Grundlagen für die Integration der Spiellogik eingebaut. Das insbesondere in Vorbereitung auf die kommenden Anpassungen und Entwicklungen die in der Projektbesprechung vom 11.12.2021 besprochen wurden. 
Konkret: 

\begin{itemize}
	\item Prüfung auf den Seiten Chars, Worldmap und Lobby ob dieser Benutzer ein aktives Spiel hat. Falls ja, wird der Benutzer auf diese Seite umgeleitet.
	\item Grundfunktion für das Beenden von einem Spiel eingebaut: Man kann nun per Klick im Spiel, das Spiel beenden.
	\item Daran anschließend eine Prüfung im laufendem Spiel, ob das Spiele beendet wurde und falls ja, Anzeige eines Endbildschirms.
\end{itemize}

Die Entwicklung der Grundlagen an diesem Tag wurde mit Fokus auf Modularisierung erledigt. Der Code der jeweiligen Funktionen wurde in einzelnen Dateien ausgelagert um Wiederverwendbarkeit und Lesbarkeit zu erhöhen. 

Die zugehörigen Commits sind insbesondere: 
\url{https://git.io/JDjKl} und 
\url{https://git.io/JDjKB}


\subsection{Entwicklung, Dienstag 19.12.2021:}

Grundlagen des Kampfsystems entsprechend der Projekt Besprechung vom 11.12.2021 (Abbildung \ref{fig:2021-12-11-Projekt-Besprechung}) sollen implementiert werden.

Vorbereitungen: 

\begin{itemize}
    \item Tabelle "GamesScenesSteps" und Verknüpfungen entfernen (Commits \url{https://git.io/JDjKz} und \url{https://git.io/JDjKg})
    \item Kampf-/Gamelog erzeugen: Darin werden alle Meldungen aus dem Spiel wie z.B. Kampftexte, Schaden, Aktionen, Systemmeldungen und alles andere denkbare angezeigt und gespeichert. Getrennt davon soll der Chat-Log dargestellt werden. Dazu werden in der Tabelle "Games" zwei neue Textfelder erzeugt (Commit \url{https://git.io/JDj63}). 
    \item Anzeige des Game-Logs auf der Spielseite. Schreiben von Nachrichten in das Gamelog als ersten Test des grundlegend umgestellten Seitenaufbaus: Es werden nur noch einzelene Elementinhalte per Websocket transportiert, nicht mehr ganze HTML-Code-Blöcke (Commit \url{https://git.io/JyeJQ}).
\end{itemize}
