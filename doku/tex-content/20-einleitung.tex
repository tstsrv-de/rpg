

\section{Einleitung}

Das Projektteam, bestehend aus Julian Schäfer, Henning Beier und Rico Pursche, beschäftigt sich im Rahmen dieser Projektarbeit des Studienfachs Web-Technologie mit der Entwicklung eines browserbasierten Rollenspiels. 
Als Projekt erschien den Studenten ein Spiel am interessantesten, denn damit kann sich jedes der Teammitglieder identifizieren. Dies erschien wichtig, um die Motivation an der Arbeit über den gesamten Zeitraum der Entwicklung hochzuhalten. Außerdem konnte sich jeder an die Zeiten erinnern als er das erste Mal eines der klassischen 8-Bit/16-Bit Spiele der Achtziger- und Neunzigerjahre gespielt hat und schnell wurde klar, dass das Spiel sich am Stil dieses auch heute noch beliebten Genres orientieren soll.

Das Spiel sollte jedoch nicht zu einfach gehalten werden, um eine gewisse Komplexität in die Entwicklung zu bringen und einer Projektarbeit des geforderten Umfanges zu entsprechen. Deshalb geben die Rahmenbedingungen vor, dass das Spiel sowohl Einzelspieler- als auch Mehrspielerkomponenten enthalten und eine Charakterentwicklung in Form von z. B. Levelaufstiegen haben soll. Außerdem ist es wünschenswert, dass alle Teilnehmer am Spiel, die für sie relevanten Informationen vom Spiel und den anderen Spielteilnehmern in Echtzeit erhalten. Dabei sollen aktuelle und geeignete Technologien zum Einsatz kommen, die den Kenntnisstand und den individuellen Fähigkeiten der einzelnen Teammitglieder entsprechen und mit denen die oben genannten Kriterien, vor allem die Online-Fähigkeit, mit Hilfe eines Browsers zu spielen, erfüllt werden können. 

So entstand die Idee ein browserbasiertes RPG im oben genannten Stil zu erschaffen und im Folgenden soll gezeigt werden, wie das Team dies im Detail umgesetzt hat. Welche Probleme bei der Spieleentwicklung auftraten und wie diese gelöst wurden, soll dabei eben so behandelt werden wie die genaue technische Umsetzung einzelner Details.
